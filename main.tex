\documentclass{article}
\usepackage[utf8]{inputenc}
\usepackage[spanish]{babel}


\begin{document}

\begin{titlepage}
    \begin{center}
        \vspace*{1cm}
            
        \Huge
        \textbf{Ideación}
            
        \vspace{0.5cm}
        \LARGE
        Informática II
            
        \vspace{1.5cm}
            
        \textbf{Juan Pablo Areiza Jiménez\\Santiago Montoya Leal}
            
        \vfill
            
        \vspace{0.8cm}
            
        \Large
        Despartamento de Ingeniería Electrónica y Telecomunicaciones\\
        Universidad de Antioquia\\
        Medellín\\
        Marzo de 2021
            
    \end{center}
\end{titlepage}
Durante la generación de ideas correspondientes al proyecto final, se pensó en la realización de un videojuego estilo run and gun, los cuales consisten en personajes que se desplazan a pie, con la posibilidad de ejecutar saltos y realizando desplazamientos tanto verticales como horizontales. Como base fundamental, el videojuego estará compuesto por varios niveles y cada uno tendrá solo un adversario (jefe), cada adversario, tendrá diferentes habilidades, atacando constantemente al jugador, el cual tendrá como objetivo esquivar todos los ataques que este arroje, además de disparar constantemente al jefe para así derrotarlo. \\

Se pretende que el videojuego desafíe las habilidades del jugador, por tal motivo serán batallas exigentes desde el principio, sin embargo, se espera que conforme avance el juego, también incremente el nivel de dificultad. Dentro de la caracterización de los jefes, se desea involucrar mecánicas especiales para cada uno, haciendo de cada combate algo totalmente diferente, manteniendo enganchado al jugador.\\

El videojuego, podrá guardar el estado del jugador en cualquier momento, sin embargo, si este fracasa, esto representará una perdida total y el jugador tendrá que comenzar desde el inicio. Como sabemos que esta dinámica puede resultar frustrante y algo compleja para la mayoría de jugadores, también se dará la posibilidad de experimentar otro modo de juego, el cual estará constituido por capítulos y el jugador podrá seleccionar cuál jugar, así, podrá disfrutar de todos los niveles del juego con una experiencia más tranquila.

\newpage

\end{document}